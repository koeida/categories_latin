\documentclass[14pt,twoside]{extbook}
\usepackage{graphicx}
\usepackage{xcolor}
\usepackage{multicol}
\usepackage{changepage}
\usepackage{lettrine}
\usepackage{marginnote}
\usepackage{etoolbox}
\usepackage{setspace}
\usepackage[font={small,it}]{caption}
\usepackage{romannum}
\usepackage{titlesec}
\usepackage{subfig}
\usepackage{wrapfig}
\usepackage{float}
\usepackage[latin.classical]{babel}
\usepackage{wallpaper}
\usepackage{fancyhdr}
\extrafloats{200}
\usepackage[paperwidth=6in, paperheight=9in]{geometry}

\newcommand{\mktitle}[1]{%
    {\begin{center}\small\textsc{\bfseries\underline{#1}}\end{center}}}


%\newgeometry{top=0.75in, bottom=1.0in, outer=1.75in, inner=0.4in, heightrounded, 
%marginparwidth=3.5cm, marginparsep=0.5cm}

\titlespacing\section{0pt}{0pt}{0pt}
\titlespacing\chapter{0pt}{0pt}{0pt}

\begin{document}

\parskip 1.5ex

\title{Categoriae}
\author{Aristotelis}

\mainmatter

\mktitle{I}

Aequivoca dīcuntur quōrum nōmen sōlum commūne est, secundum nōmen vērō
substantiae ratiō dīversa, ut animal homō et quod pingitur. Hōrum enim
sōlum nōmen commūne est, secundum nōmen vērō substantiae ratiō
dīversa; sī enim quis assignet quid est utrīque eōrum quō sint
animālia, propriam assignābit utriusque ratiōnem.

Ūnivoca vērō dīcuntur quōrum et nōmen commūne est et secundum nōmen
eadem substantiae ratiō, ut animal homō atque bōs. Commūnī enim nōmine
utrīque animālia nuncupantur, et est ratiō substantiae eadem; sī quis
enim assignet utriusque ratiōnem, quid utrīque sit quō sint animālia,
eandem assignābit ratiōnem.

\newpage

\mktitle{I}

Things are said to be named `equivocally' when, though they have a common name, the definition corresponding with the name differs for each. Thus, a real man and a figure in a picture can both lay claim to the name `animal'; yet these are equivocally so named, for, though they have a common name, the definition corresponding with the name differs for each. For should any one define in what sense each is an animal, his definition in the one case will be appropriate to that case only.

On the other hand, things are said to be named `univocally' which have both the name and the definition answering to the name in common. A man and an ox are both `animal', and these are univocally so named, inasmuch as not only the name, but also the definition, is the same in both cases: for if a man should state in what sense each is an animal, the statement in the one case would be identical with that in the other.

\newpage

Dēnōminātīva vērō dīcuntur quaecumque ab aliquō, solō differentia
cāsū, secundum nōmen habent appellātiōnem, ut ā grammaticā grammaticus
et ā fortitūdine fortis.

\newpage

Things are said to be named `derivatively', which derive their name from some other name, but differ from it in termination. Thus the grammarian derives his name from the word `grammar', and the courageous man from the word `courage'.

\newpage

\mktitle{II}

Eōrum quae dīcuntur alia quidem secundum complexiōnem dīcuntur, alia
vērō sine complexiōne. Et ea quae secundum complexiōnem dīcuntur sunt
ut homō currit, homō vincit; ea vērō quae sine complexiōne, ut homō,
bōs, currit, vincit.

Eōrum quae sunt alia dē subiectō quōdam dīcuntur, in subiectō vērō
nūllō sunt, ut homō dē subiectō quidem dīcitur aliquō homine, in
subiectō vērō nūllō est; 

(In subiectō autem esse dīcō quod, cum in
aliquō sit nōn sīcut quaedam pars, impossibile est esse sine eō in quō
est.)

Alia autem in subiectō quidem sunt, dē
subiectō vērō nūllō dīcuntur.  Ut quaedam grammatica in subiectō quidem est in animā, dē
subiectō vērō nūllō dīcitur, et quoddam album in subiectō est in
corpore (omnis enim color in corpore est).\newpage

\mktitle{II}

Forms of speech are either simple or composite. Examples of the latter are such expressions as `the man runs', `the man wins'; of the former `man', `ox', `runs', `wins'.

Of things themselves some are predicable of a subject, and are never present in a subject. Thus `man' is predicable of the individual man, and is never present in a subject.

(By being `present in a subject' I do not mean present as parts are present in a whole, but being incapable of existence apart from the said subject.)

Some things, again, are present in a subject, but are never predicable of a subject. For instance, a certain point of grammatical knowledge is present in the mind, but is not predicable of any subject; or again, a certain whiteness may be present in the body (for colour requires a material basis).\newpage

Alia vērō et dē subiectō
dīcuntur et in subiectō sunt, ut scientiā in subiectō quidem est in
animā, dē subiectō vērō dīcitur dē grammaticā. 

Alia vērō neque in
subiectō sunt neque dē subiectō dīcuntur, ut aliquis homō vel aliquis
equus; nihil enim hōrum neque in subiectō est neque dē subiectō dīcitur. Simpliciter autem quae sunt indīvidua et numerō singulāria
nūllō dē subiectō dīcuntur, in subiectō autem nihil ea prohibet esse;
quaedam enim grammatica in subiectō est.\newpage

Other things, again, are both predicable of a subject and present in a subject. Thus while knowledge is present in the human mind, it is predicable of grammar.

There is, lastly, a class of things which are neither present in a subject nor predicable of a subject, such as the individual man or the individual horse. But, to speak more generally, that which is individual and has the character of a unit is never predicable of a subject. Yet in some cases there is nothing to prevent such being present in a subject. Thus a certain point of grammatical knowledge is present in a subject.\newpage

\mktitle{III}

Quandō alterum dē alterō praedicātur ut dē subiectō, quaecumque dē
eō quod praedīcātur dīcuntur, omnia etiam dē subiectō dicentur, ut
homō dē quōdam homine praedicātur, animal vērō dē homine, ergō et dē
quōdam homine animal praedicābitur; quīdam enim homō et homō est et
animal.

Dīversōrum generum et nōn subalternatim positōrum dīversae secundum
speciem et differentiae sunt, ut animālis et scientiae; animālis
quidem differentiae sunt ut gressibile et volātile et bipēs, scientiae
vērō nūlla hārum est; neque enim scientia ab scientiā differt in eō
quod bipēs est.

Subalternorum vērō generum nihil prohibet eāsdem esse
differentiās; superiōra enim dē subterioribus generibus praedicantur,
quārē quaecumque praedicātī differentiae fuerint, eaedem erunt etiam
subiectī.\newpage

\mktitle{III}

When one thing is predicated of another, all that which is predicable of the predicate will be predicable also of the subject. Thus, `man' is predicated of the individual man; but `animal' is predicated of `man'; it will, therefore, be predicable of the individual man also: for the individual man is both `man' and `animal'.

If genera are different and co-ordinate, their differentiae are themselves different in kind. Take as an instance the genus `animal' and the genus `knowledge'. `With feet', `two-footed', `winged', `aquatic', are differentiae of `animal'; the species of knowledge are not distinguished by the same differentiae. One species of knowledge does not differ from another in being `two-footed'.

But where one genus is subordinate to another, there is nothing to prevent their having the same differentiae: for the greater class is predicated of the lesser, so that all the differentiae of the predicate will be differentiae also of the subject.  \newpage

\mktitle{IV}

Eōrum quae secundum nūllam complexiōnem dīcuntur singulum aut
substantiam significat aut quantitātem aut quālitātem aut ad aliquid
aut ubi aut quandō aut situm aut habitum aut facere aut patī. Est
autem substantia quidem ut figurātim dīcātur ut homō, equus; quantitās
ut bicubitum, tricubitum; quālitās ut album; ad aliquid ut duplum,
maius; ubi vērō ut in Lycīō; quandō autem ut heri; situs vērō ut
sedet, iacet; habēre autem ut calciātus, armātus; facere vērō ut
secāre, ūrere; patī vērō ut secārī, ūrī.

Singula igitur eōrum quae
dicta sunt ipsa quidem secundum sē in nūllā affirmātiōne dīcuntur,
hōrum autem ad sē invicem complexiōne affirmātiō fit. Vidētur enim
omnis affirmātiō vel falsa esse vel vēra; eōrum autem quae secundum
nūllam complexiōnem dīcuntur neque vērum quicquam neque falsum est, ut
homō, album, currit.\newpage

\mktitle{IV}

Expressions which are in no way composite signify substance, quantity, quality, relation, place, time, position, state, action, or affection. To sketch my meaning roughly, examples of substance are `man' or `the horse', of quantity, such terms as `two cubits long' or `three cubits long', of quality, such attributes as `white', `grammatical'. `Double', `half', `greater', fall under the category of relation; `in a the market place', `in the Lyceum', under that of place; `yesterday', `last year', under that of time. `Lying', `sitting', are terms indicating position, `shod', `armed', state; `to lance', `to cauterize', action; `to be lanced', `to be cauterized', affection.

No one of these terms, in and by itself, involves an affirmation; it is by the combination of such terms that positive or negative statements arise. For every assertion must, as is admitted, be either true or false, whereas expressions which are not in any way composite such as `man', `white', `runs', `wins', cannot be either true or false.\newpage

\mktitle{V: De Substantia}

Substantia autem est, quae proprie et principaliter et maxime dicitur,
quae neque de subiecto praedicatur neque in subiecto est, ut aliqui
homo uel aliqui equus. Secundae autem substantiae dicuntur, in quibus
speciebus illae quae principaliter substantiae dicuntur insunt, hae et
harum specierum genera; ut aliquis homo in specie quidem est in
homine, genus uero speciei animal est; secundae ergo substantiae
dicuntur, ut est homo atque animal.

Manifestum est autem ex his quae
dicta sunt quoniam eorum quae de subiecto dicuntur necesse est et
nomen et rationem de subiecto praedicari, ut homo de subiecto dicitur
aliquo homine, et praedicatur nomen; namque hominem de aliquo homine
praedicabis. Ratio quoque hominis de aliquo homine praedicabitur;
quidam enim homo et homo est. Quare et nomen et ratio praedicabitur de
subiecto.\newpage

\mktitle{V: On Substance}

Substance, in the truest and primary and most definite sense of the word, is that which is neither predicable of a subject nor present in a subject; for instance, the individual man or horse. But in a secondary sense those things are called substances within which, as species, the primary substances are included; also those which, as genera, include the species. For instance, the individual man is included in the species 'man', and the genus to which the species belongs is 'animal'; these, therefore-that is to say, the species 'man' and the genus 'animal,-are termed secondary substances.

It is plain from what has been said that both the name and the definition of the predicate must be predicable of the subject. For instance, 'man' is predicted of the individual man. Now in this case the name of the species man' is applied to the individual, for we use the term 'man' in describing the individual; and the definition of 'man' will also be predicated of the individual man, for the individual man is both man and animal. Thus, both the name and the definition of the species are predicable of the individual.\newpage

Eorum uero quae sunt in subiecto, in pluribus quidem neque
nomen de subiecto neque ratio praedicatur, in quibusdam uero nomen
quidem nihil prohibet praedicari, rationem uero impossibile est; ut
album, cum in subiecto sit corpore, praedicatur de subiecto (dicitur
enim corpus album), ratio uero albi numquam de corpore
praedicabitur. Caetera uero omnia aut de subiectis dicuntur primis
substantiis aut in eisdem subiectis sunt. Hoc autem manifestum est ex
his quae singulatim proferuntur; ut animal de homine praedicatur,
quare et de aliquo homine praedicabitur; nam si de nullo aliquorum
hominum diceretur, nec de ipso homine praedicaretur omnino. Rursus
color in corpore est; ergo et in aliquo corpore; nam si in nullo esset
corporum singulorum, nec in corpore esset omnino. Quocirca caetera
omnia aut de subiectis primis substantiis dicuntur aut in subiectis
ipsis sunt. Si ergo primae substantiae non sunt, impossibile est
aliquid esse caeterorum. Omnia enim alia aut de ipsis subiectis
dicuntur aut in subiectis ipsis sunt; quare, si primae substantiae non
sunt, impossibile est aliquid esse caeterorum.

\end{document}
